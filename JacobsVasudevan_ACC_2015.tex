\documentclass[letterpaper, 10 pt, conference]{ieeeconf}
\IEEEoverridecommandlockouts  % for thanks command
\overrideIEEEmargins  % Needed to meet printer requirements.
\usepackage{amsmath} % assumes amsmath package installed
\usepackage{amssymb}  % assumes amsmath package installed
\usepackage{todonotes}

% NEW COMMANDS
\newcommand{\hoj}[1]{\todo[inline,color=blue!20]{#1}}
\newcommand{\ram}[1]{\todo[inline,color=red!20]{#1}}
\newcommand{\R}{\mathbb{R}}

% NEW THEOREM ENVIRONMENTS
\newtheorem{thm}{Theorem}[section]
\newtheorem{defn}[thm]{Definition}
\newtheorem{prop}[thm]{Proposition}
\newtheorem{cor}[thm]{Corollary}
\newtheorem{lem}[thm]{Lemma}

% MATH OPERATORS
\DeclareMathOperator{\Diff}{Diff}
\DeclareMathOperator{\Fr}{Fr}
\DeclareMathOperator{\GL}{GL}
\DeclareMathOperator{\Dens}{Dens}

\title{\LARGE \bf
  An advection scheme for probabilty densities on manifolds
}


\author{Henry O. Jacobs and Ram Vasudevan}% <-this % stops a space


\begin{document}

\maketitle
\thispagestyle{empty}
\pagestyle{empty}


%%%%%%%%%%%%%%%%%%%%%%%%%%%%%%%%%%%%%%%%%%%%%%%%%%%%%%%%%%%%%%%%%%%%%%%%%%%%%%%%
\begin{abstract}
Half densities, advection, blah blah blah...
WTF, is this why they call it an ``abstract''?!

\end{abstract}


%%%%%%%%%%%%%%%%%%%%%%%%%%%%%%%%%%%%%%%%%%%%%%%%%%%%%%%%%%%%%%%%%%%%%%%%%%%%%%%%
\section{Introduction}
  Let $p_0$ be a probability density on $\R^n$
  and let $X$ be a vector-field on $\R^n$.
  We denote the flow of $X$
  by $\Phi_X^t$.
  We can define the time-dependent probability
  density $p$ by
  \begin{align}
    p( x ; t) = 
    | \det( D\Phi_X^t(x) ) |^{-1} p_0( \Phi_X^t(x) ), \label{eq:push_forward}
  \end{align}
  where $D\Phi_X^t(x)$ is the Jacobian matrix of
  $\Phi_X^t$ at the point $x \in \R^n$.
  The density $p(x;t)$ represents how the density
  $p_0(x)$ transforms under the flow of $X$.
  We observe that $p(x;0) = p_0(x)$ and, upon taking
  a time derivative,
  \begin{align}
    \partial_t p (x;t) + \partial_i (X^i p)(x) = 0 \label{eq:Louiville}
  \end{align}
  for all $x \in \R^n$.
  Equation \eqref{eq:Louiville} is known as the \emph{Louiville equation}.
  Using \eqref{eq:push_forward} to compute $p$ is typically difficult
  because $\Phi_X^t$ is typically a non-linear map with no closed
  form expression.
  To compute $p( x ; t)$, it is often easier
  to numerically solve \eqref{eq:Louiville}
  (a first order linear evolution PDE)
  with the initial condition $p_0$.
  
  In this article we will study the advection of probability
  densities on manifolds.
  The corresponding advection equation is a linear
  evolution PDE, whos description in local coordinates is 
  given by \eqref{eq:Louiville}.

\subsection{A naive psuedo-spectral method}
\label{sec:naive}
  Let $f_0,f_1,f_2,\dots$ serve as an
  orthonormal Hilbert basis
  for $L_2(\R^n)$ (e.g. products of sinusoids).
  Let $\rho(x,t)$ be the solution to \eqref{eq:Louiville}.
  Assuming $\rho(\cdot,t) \in L^2$ and $\partial_i X^i \in L^{\infty}$,
  we can define the scalars $\rho_j(t), A^k_j \in \R$
  by $\rho_j(t) = \langle \rho_t( \cdot ) , f_j \rangle_{L^2}$
  and $A^k_j = \langle (\partial_i X^i) f_j , f_k \rangle$.
  Then $\rho_j(t)$ satisfies the infinite-dimensional linear
  ordinary differential equation
  $
    \dot{\rho}_j = - \sum_{k=0}^{\infty}A^k_j \rho_k
  $.
  Moreover, $\rho(x,t) = \rho_j(t) f_j(x)$.
  We can truncate this system at some finite $N \in \mathbb{N}$
  to obtain an $N$-dimensional linear ordinary differential equation
  $\dot{\rho}_{N,j} = -\sum_{k=0}^{N} A^k_j \rho_{N,j}$ for $j=0,\dots,N$.
  It is notable that if $f_0,f_1,\dots$ is a Haar basis, then the matrix
  $A^k_j$ is equivalent to a transfer operator.
  Under the right circumstances,
  this method converges as $N \to \infty$
  \cite{FroylandJungeKoltai2013}.

  However, for $N < \infty$, there is no guarantee that the reconstructed
  density $\tilde{\rho}(x,t) = \sum_{j=0}^N \rho_{N,j}(t) f_j(x)$
  is non-negative.
  Generically $\tilde{\rho}(x,t)$ will take on both signs, in
  contrast with the exact solution $\rho(x,t)$.
  Moreover, the total mass $\tilde{m}(t) = \int \tilde{\rho}(x,t)dx$
  may fluctuate, also in contrast with the exact total mass
  $m = \int \rho(x,t) dx$.
  When dealing with manifolds of even moderate dimensions
  (e.g. $3$) it is important for a method to behave well at finite $N$'s
  because it is infeasible to finely resolve along each dimension.

\subsection{Main contributions}
  In this paper we will present a method for advection of
  probability densities on manifolds.
  This method will yield reconstructed probability measures
  which are non-negative and mass conserving at any finite resolution.

\section{Mathematical preliminaries}
\label{sec:math}
  Throughout this section we will have the following
  setup.  Let $M$ be a smooth manifold.
  We denote the tangent bundle of $M$ by $TM$.
  Given any $C^1$ function $f:M \to N$,
  we denote the tangent lift of $f$ by $Tf:TM \to TN$.

\subsection{Densities}
  A smooth density on $M$ is a smooth means of
  assigning real numbers to measureable sets.
  Hueristically, it is a map which
  takes an infinitesiaml box (or volume element)
  on a manifold as input, and outputs the infinitesimal ``size''
  of the box (a real number, possibly negative).
  Therefore, in order to discuss densities,
  we must first mathematize the notion of an ``infinitesimal box.''
  This motivates our introduction of \emph{frames}.
  \begin{defn}
  \label{eq:frame_bundle}
    Given a manifold $M$, and a point $x \in M$,
    a \emph{frame at $x$} is a basis on the tangent space $T_x M$.
    We denote the set of frames at $x$ by $\Fr_x M$.
    The frame bundle is $\Fr M = \cup_{x \in M} \Fr_x M$.
  \end{defn}

  \begin{prop}
    There is a natural transitive
    action of $\GL(n)$ on each fiber of $\Fr M$.
  \end{prop}

  \begin{proof}
    Let $e = \{ e^1,\dots,e^n \} \in \Fr_x M$ for some $x \in M$.
    For each $A \in \GL(n)$ define the left action
    \begin{align*}
      A \cdot (e^1,\dots,e^n) := (A^j_1 e^j , \dots, A_n^j e^j ). 
    \end{align*}
    By inspection this actions is free and transitive.
    \footnote{This makes $\Fr M$ a $\GL(n)$ principal bundle over $M$.}
  \end{proof}

  Now that we understand frames (i.e. infinitesimal boxes)
  sufficiently well, we may introduce the notion of \emph{densities}.

  \begin{defn}
    \label{eq:density}
    Let $\alpha > 0$.
    An $\alpha$-\emph{density} on a manifold, $M$, is a map
    $\rho:\Fr M \to \R$ such that
    for any frame $e \in \Fr M$, and
    $
      \rho( A \cdot e ) =  | \det(A) |^\alpha \rho(e).
    $
    We denote the space of densities by $\Dens^\alpha(M)$.
    A $1$-density is simply called a \emph{density}, and
    so we denote $\Dens^1(M)$ by $\Dens(M)$.
    The integral of a density is defined via the same construction as that
    for $n$-forms \cite[\S 14]{Lee2006}.
    A \emph{mass density} is a density which is non-negative.
    and it is called a \emph{probability density} if its integral is
    unity.
  \end{defn}

  One can observe immediately that $\Dens^\alpha(M)$ is a vector-space.
  Despite this commomanilty with tensors,
  $1$-densities are \emph{not} tensors.
  Densities are very close in spirit to $n$-forms,
  but unlike $n$-forms, densities are \emph{non-oriented}
  due to the use of ``$|\det(A)|$'' rather than ``$\det(A)$'' 
  in the definition.
  Therefore, a density will not flip signs under a change of basis.
  This allows for the integral of a density to be well defined
  on non-orientable manifolds as well \cite[Ch. 14]{Lee2006}.

  \begin{prop}[Appendix A \cite{BatesWeinstein1997}]
    Let $\psi_1,\psi_2 \in \Dens^{1/2}(M)$.
    The function, $\psi_1 \psi_2$, obtained by
    scalar multiplication is a $1$-density.
    The pairing
    $
    \langle \psi_1, \psi_2 \rangle := \int_M \psi_1 \psi_2 
    $
    is a real inner-product.
    Finally, for any density $\rho$, the functions $\pm\sqrt{\rho}$ are
    $\frac{1}{2}$-densities.
  \end{prop}
  \begin{proof}
    Let $e \in \Fr M$ and $A \in \GL(n)$.
    We observe $\psi_1(A \cdot e) \psi_2(A \cdot e) = |\det(A) | \psi_1(e) \psi_2(e)$.  Thus $\psi_1 \psi_2 \in \Dens(M)$.
    Conversely $\pm \sqrt{\rho( A \cdot e)} = \pm | \det(A) |^{1/2} \sqrt{ \rho(e)}$. So $\pm \sqrt{\rho} \in \Dens^{1/2}(M)$.
    Finally, if $\psi \neq 0$ we see that $\| \psi \|^2 := \langle \psi , \psi \rangle \neq 0$.
    Thus $\langle \cdot , \cdot \rangle$ is weakly non-degenerate
    and defines an inner-product on $\Dens^{1/2}(M)$.
  \end{proof}

  Note that for any diffeomophism $\Phi \in \Diff(M)$,
  there is a map $\Fr(\Phi) : \Fr M \to \Fr M$ defined by
  \begin{align*}
    \Fr(\Phi) \cdot (e_1,\dots,e_n) = 
    (T\Phi \cdot e_1,\dots,T\Phi \cdot e_n).
  \end{align*}
  This defines the \emph{pull-back} of an $\alpha$-density
  $\nu \in \Dens^\alpha(M)$ by $\Phi^* \nu := \nu \circ \Fr(\Phi)$.
  \begin{prop} \label{prop:isom}
    Let $\Phi \in \Diff(M)$.
    The transformation ``$\psi \mapsto \Phi^* \psi$'' for
    $\psi \in \Dens^{1/2}(M)$ is an isometry.
  \end{prop}
  \begin{proof}
    Let $\psi_1,\psi_2 \in \Dens^{1/2}(M)$ and observe
    \begin{align*}
      \langle \Phi^* \psi_1, \Phi^* \psi_2 \rangle
      = \int \Phi^*( \psi_1 \psi_2)
      = \int \psi_1 \psi_2
      = \langle \psi_1, \psi_2 \rangle,
    \end{align*}
    where the equivalence of the integrals follows
    from \cite[Proposition 14.32(c)]{Lee2006}.
  \end{proof}
  
  Given a vector field $X \in \mathfrak{X}(M)$
  we can denote the flow by $\Phi^t_X \in \Diff(M)$
  and define the Lie-derivative of an alpha density by
  $
    \pounds_X[ \nu ] := \left. \frac{d}{dt} \right|_{t=0} (\Phi_X^t)^* \nu.
  $
  This yields the following corollary to proposition \ref{prop:isom}.
  \begin{cor}
    For any $X \in \mathfrak{X}(M)$, $\pounds_X[ {}\cdot{} ]$
    is an anti-symmetric linear operator on $\Dens^{1/2}(M)$.
  \end{cor}
  With the Lie-derivative defined we can write the advection
  PDE for densities as
  \begin{align}
    \partial_t \rho + \pounds_X[\rho] = 0 \label{eq:advection},
  \end{align}
  for a time-dependent density $\rho(t) \in \Dens(M)$
  which is advected infinitesimally by the vector field 
  $X \in \mathfrak{X}(M)$.
  In local coordinates, \eqref{eq:advection} is written the same
  as \eqref{eq:Louiville}.

  Moreover, for a time-dependent half-density $\psi(t) \in \Dens^{1/2}(M)$
  we may consider the advection equation
  \begin{align}
    \partial_t \psi + \pounds_X[\psi] = 0 \label{eq:advection2}
  \end{align}
  which is written in local coordinates as
  \begin{align*}
    \partial_t \psi(x) 
    + X^i(x) \partial_i \psi(x) 
    + \frac{1}{2} (\partial_i X^i)(x) \psi(x) = 0.
  \end{align*}
  
  \begin{thm} \label{thm:advection}
    Let $\rho(t) \in \Dens(M)$ be a time-dependent probability density
    and let $\psi \in \Dens^{1/2}(M)$ be such that $\rho = \psi^2$.
    Assume that $\rho$ is $C^1$.
    Let $X \in \mathfrak{X}(M)$.
    The following are equivalent:
    \begin{enumerate}
      \item $\rho$ satisfies the advection equation \eqref{eq:advection}.
      \item $\psi$ satisfies the advection equation \eqref{eq:advection2}.
    \end{enumerate}
  \end{thm}
  \begin{proof}
    Let $\psi$ satisfy \eqref{eq:advection2}.
    We find
    \begin{align*}
      \partial_t (\psi^2) &= 2 \partial_t\psi \cdot \psi
      =-2 \pounds_X[\psi] \psi \\
      &= - 2 \left[ \left.\frac{d}{dt}\right|_{t=0}
         (\Phi_X^t)^* \psi \right] \cdot \psi \\
      &= - \left. \frac{d}{dt} \right|_{t=0}
        \left[ (\Phi_X^t)^* \psi \cdot (\Phi_X^t)^* \psi \right]\\
      &= - \left. \frac{d}{dt} \right|_{t=0}
        \left[ (\Phi_X^t)^* (\psi^2) \right] 
      = - \pounds_X[\psi^2].
    \end{align*}
    Therefore $\rho = \psi^2$ satisfies \eqref{eq:advection}.
    Conversely, if $\rho$ satisfies \eqref{eq:advection}
    and $\psi^2 = \rho$ then
    \begin{align*}
      \partial_t (\psi^2) = - \pounds_X[\psi^2] = - \pounds_X[\psi] \psi.
    \end{align*}
    Moreover, the right hand side is $2 (\partial_t \psi) \psi$.
    We can divide both side by $\psi$ at any point where $\psi(x) \neq 0$.
    By continuity of $\partial_t \rho$ and $\partial_t \psi$ we can
    verify \eqref{eq:advection2} on the entire support of $\psi$
    (which is also the support of $\rho$).
    Outside the support it is neccesarily the case that
    $\rho = 0$ and  $\partial_i \rho = 0$.
    We observe that $\pounds_X[\rho] = 0$ as well.
    In this case $\pounds_X[\psi] = 0$ by the same argument.
    So we've verified \eqref{eq:advection2} on the entire domain.
  \end{proof}

  We will use theorem \ref{thm:advection} later to justify building
  a numerical scheme to solve \eqref{eq:advection2} in lieu of solving 
  \eqref{eq:advection}.

\subsection{Euclidean realizations}
\label{sec:euclidean}
We would like to apply wavelet theory later for the purpose of the
error bounds and sparsity structure they will bring us.
However, the notion of wavelets on manifolds is still in development,
and virtually all of the available packages are for wavelets on
Euclidean spaces.
This motivates us to prove the following theorem.

\begin{thm} \label{thm:global_chart}
  Let $M$ be a manifold and let
  $\phi:U \subset M \to V \subset \mathbb{R}^n$
  be a chart such that $\bar{U} = M$.
  Then the subspace of half-densities over $V$ given by
  $(i_U \circ \phi^{-1})^* \Dens^{1/2}(M)$ is isomorphic
  to $\Dens^{1/2}(M)$.
\end{thm}
\begin{proof}
  Elements of $(i_U \circ \phi^{-1})^* \Dens^{1/2}(M)$
  are half-densities on $V \subset \mathbb{R}^n$ given by
  $ \left. \psi \right|_{U} \circ \Fr(\phi)$ for some
  $\psi \in \Dens^{1/2}(M)$.
  Thus the map $\psi \mapsto \psi|_U \circ \Fr(\phi)$
  provides a surjection.
  We wish to show that this map is also an injection.
  Specifically this means, given 
  $\left. \psi \right|_{U} \circ \Fr(\phi)$ in
  $(i_U \circ \phi^{-1})^*\Dens^{1/2}(M)$ the half-density $\psi$ is
  uniquely determined.
  To prove this, assume that there exists another half-density,
  $\tilde{\psi}$, such that
  $\tilde{\psi}|_U \circ \phi^{-1} = \psi|_U \circ \phi^{-1}$.
  For any $x \in V$ and ${\bf e}_x \in \Fr_x(V)$ we observe that
  \begin{align*}
    \tilde{\psi}( \Fr_x(\varphi) \cdot {\bf e}_x) =
    \psi( \Fr_x( \varphi) \cdot {\bf e}_x).
  \end{align*}
  Applying this over all $x \in V$ matches $\psi$ and $\tilde{\psi}$ on
  the subset $U \subset M$.
  Therefore, the only place where $\psi$ and $\tilde{\psi}$ may differ 
  is on $M \ U$.
  As $U$ is dense, $M$ is compact, and $\psi|_U$ is continous,
  the Tietze extension theorem yields a unique $C^0$ extension
  $\psi|_U$ to $M$.  This unique extension must be none other
  that $\psi$ itself, and so $\psi = \tilde{\psi}$.
\end{proof}

  Theorem \ref{thm:global_chart} is useful because in translates
  PDEs on manifolds into PDEs on subsets of $\mathbb{R}^n$.
  One concern is the assumption that there exists a chart which
  is dense.  However, if $M$ is compact such a chart is guaranteed
  to exist.  This follows from the fact that any manifold can have
  a Riemannian structures placed upon it, and compact Riemannian
  manifolds are geodesically complete.
  The Riemannian exponential map at a point in the manifold
  yields a star-shaped dense chart whose boundary is the cut-locus
  of the given point \cite{Sakai1996}.
  The existence of such a chart in the non-compact case is unclear
  at this time.

\begin{defn}
  Let $M$ be a compact connected manifold (with or without boundary), and let $\phi:U \subset M \to V \subset \mathbb{R}^n$ be a chart such that $U$ is dense in $M$.
  Let $W_s \subset C^k(\bar{V})$ be a nested sequence of function spaces
  with respect to the paramenter $s \in S$.
  We call the pair $(\phi, W_s)$ a \emph{Euclidean realization of $M$}
  if $\phi^* i_V^*W_s \subset i_{U}^* C^k(M)$ for each $s$ and $\cup_s \phi^*i_V^*W_s$ is dense in $i_{U}^*C^k(M)$.
\end{defn}

The constraint ``$\phi^*i_V^*W_s \subset i_U^* C^k(M)$'' enforces that each of the spaces $W_s$ models the function space $i_U^* C^k(M)$ and nothing more.
By theorem \ref{thm:global_chart}, $i_U^* C^k(M) = C^k(M)$, and thus $W_s$ models a subspace of the functions on $C^k(M)$.
The second constraint ensures that elements of $i_U^* C^k(M)$ can be approximated by $W_s$ for some $s \in S$.

\section{An advection scheme}
\label{sec:scheme}


\subsection{Convergence}
\label{sec:convergence}

\subsection{Properties}
\label{sec:properties}
Some propeties:  Positivity,
Anti-symmetry (diagnolizable, easily exponentiated),
mass conserving.


\section{Examples}
\label{sec:Examples}

\subsection{Comparison with Tranfer-operator methods}
Consider the ODE $\dot{\theta} = \cos(\theta)$ on the unit circle, $S^1$.

\hoj{Compare two low res computations.
  One with transfer operators, the other with our method.}

\section{Future work}

\bibliographystyle{IEEEtran}
\bibliography{/Users/hoj201/Dropbox/hoj_2014}

\end{document}
