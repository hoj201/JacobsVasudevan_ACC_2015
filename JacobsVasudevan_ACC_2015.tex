\documentclass[letterpaper, 12 pt]{amsart}
\usepackage[margin=1.5in]{geometry}
\usepackage{amsmath} % assumes amsmath package installed
\usepackage{amssymb}  % assumes amsmath package installed
\usepackage{todonotes}
\usepackage{amsaddr}
%\usepackage[nomarkers,figuresonly]{endfloat}

% NEW COMMANDS
\newcommand{\hoj}[1]{\todo[inline,color=blue!20]{#1}}
\newcommand{\ram}[1]{\todo[inline,color=red!20]{#1}}
\newcommand{\R}{\mathbb{R}}

% NEW THEOREM ENVIRONMENTS
\newtheorem{thm}{Theorem}[section]
\newtheorem{defn}[thm]{Definition}
\newtheorem{prop}[thm]{Proposition}
\newtheorem{cor}[thm]{Corollary}
\newtheorem{lem}[thm]{Lemma}
\newtheorem{ass}[thm]{Assumption}


% MATH OPERATORS
\DeclareMathOperator{\Diff}{Diff}
\DeclareMathOperator{\Fr}{Fr}
\DeclareMathOperator{\GL}{GL}
\DeclareMathOperator{\Dens}{Dens}
\DeclareMathOperator{\pr}{pr}
\DeclareMathOperator{\arccot}{arccot}

\title{
  A 
  multiscale
  advection scheme for
  densities on manifolds
}

\author{Henry O. Jacobs}
\address{Research Associate in Mathematics\\
180 Queen's Gate Rd, London, United Kingdom, SW7 2AZ \\}

\author{Ram Vasudevan}
\address{Assistant Professor in Mechanical Engineering\\
G058 AL (W.E. Lay Auto Lab)\\
1231 Beal , Ann Arbor, MI 48109-2133}
\email{h.jacobs@imperial.ac.uk,ramv@umich.edu}

\begin{document}

\maketitle

\begin{abstract}
  The task of computing a probability density advected by an 
  dynamical system may be viewed as an inifinite dimensional problem
  on the positive cone of the unsigned densities.
  Existing schemes exhibit numerical artifacts such as
  ``negative probabilities''.
  In this article we present a method
  which preserves the positivity of probability densities
  at arbitrarily low resolutions.
  Moreover, we can use a single dense chart to transport the machinary of wavelet analysis to a manifolds.
  This allows us to avoid the use of transition maps between multiple charts, and to implement our method on a variety of non-Euclidean spaces at multiple length scales with existing wavelet algorithms designed on $\R^n$.
\end{abstract}


%%%%%%%%%%%%%%%%%%%%%%%%%%%%%%%%%%%%%%%%%%%%%%%%%%%%%%%%%%%%%%%%%%%%%%%%%%%%%%%%
\section{Introduction}
  The task of advecting a probability density presents itself
  in a variety of scenarios.
  Engineers are often presented
  with dynamical systems and incomplete knowledge
  of the initial conditions.
  If there is some region, $S$, of state-space which is ``dangerous''
  he or she may wish to compute the probability of landing
  in this dangerous region at time $T$.
  If the initial condition is given in the form of a
  probability density $p(\cdot \mid t=0)$
  the probability of landing in $S$ at time $T$
  is $P( x \in S \mid t = T)  = \int_S p( dx | t=T )$
  where $p(dx|t=T)$ is the advected probability density at time $T$.
  In order to compute $p(dx\mid t=T)$
  one must advect $p(dx\mid t=0)$ under the flow of the dynamics.
  
  Computing $p(dx \mid t=T)$ is expensive, as in the solution of any
  functional evolution equation.
  This cost is exponentially exacerbated by the curse of dimensionality.
  Roughly speaking, we run into data storage problems if we try to resolve
  our numerical mesh.
  Thus we seek an algorithm which has good behavior at moderate resolutions.
  In this paper we will present an algorithm which preserves the positivity
  of the space of probability densities at arbitrary resolutions.

\subsection{Background material}
  Let $p_0$ be a probability density on $\R^n$
  and let $X$ be a vector-field on $\R^n$.
  We denote the flow of $X$
  by $\Phi_X^t$.
  We can define the time-dependent probability
  density $p$ by
  \begin{align}
    p( x ; t) = 
     \det\left[ (D\Phi_X^t)^{-1}(x) \right] p_0\left( \left[\Phi_X^t\right]^{-1}(x) \right), \label{eq:push_forward}
  \end{align}
  where $D\Phi_X^t(x)$ is the Jacobian matrix of
  $\Phi_X^t$ at the point $x \in \R^n$.
  The density $p(x;t)$ represents how the density
  $p_0(x)$ transforms under the flow of $X$.
  We observe that $p(x;0) = p_0(x)$ and, upon taking
  a time derivative,
  \begin{align}
    \partial_t p (x;t) + \partial_i (X^i p)(x) = 0 \label{eq:Louiville}
  \end{align}
  for all $x \in \R^n$.
  Equation \eqref{eq:Louiville} is known as the \emph{Louiville equation} (see \cite{LasotaMackey1994}).
  Using \eqref{eq:push_forward} to compute $p$ is difficult
  because $\Phi_X^t$ is typically a non-linear map with no closed
  form expression.
  To compute $p( x ; t)$, it is often easier
  to numerically solve \eqref{eq:Louiville}
  (a first order linear evolution PDE)
  with the initial condition $p_0$.
  On a manifold $M$, the advection equation is a linear
  evolution PDE involving the Lie-derivative
  where the description in a local coordinate chart is
  given by \eqref{eq:Louiville}.
  This will be addressed in \S \ref{sec:math} (see \eqref{eq:advection}).

\subsection{A naive pseudo-spectral method}
\label{sec:naive}
  In this section we will present the simplest type of
  spectral discretization of \eqref{eq:Louiville}.
  Let $f^0,f^1,f^2,\dots \in L^2(\R^n)$ serve as an
  orthonormal Hilbert basis (e.g. a Fourier basis).
  Let $\rho(x,t)$ be the solution to \eqref{eq:Louiville}.
  Assuming $\rho(\cdot,t) \in L^2$ and $\partial_i X^i \in L^{\infty}$,
  we can define the scalars $\rho_j(t), A^k_j \in \R$
  by $\rho_j(t) = \langle \rho(\cdot ,t) , f^j \rangle_{L^2(\R^n)}$
  and $A^k_j = \langle (\partial_i X^i) f^j , f^k \rangle_{L^2(\R^n)}$.
  Then $\rho_j(t)$ satisfies the infinite-dimensional linear
  ordinary differential equation
  $
    \dot{\rho}_j = - \sum_{k=0}^{\infty}A^k_j \rho_k
  $.
  Moreover, $\rho(x,t) = \rho_j(t) f_j(x)$.
  We can truncate this system at some finite $N \in \mathbb{N}$
  to obtain an $N$-dimensional linear ordinary differential equation
  $\dot{\rho}_{N,j} = -\sum_{k=0}^{N} A^k_j \rho_{N,j}$ for $j=0,\dots,N$.
  It is notable that if $f_0,f_1,\dots$ is a Haar basis, then
  at finite resolution, this algorithm is equivalent to 
  partitioning the space into cells and the basis is equivalent
  to a set of indicator functions on the cells.
  The matrix $A^k_j$ is then simply a matrix of fluxes 
  known as a transfer operator.
  Under the right circumstances,
  this method converges as the cell width approaches $0$ and $N \to \infty$
  \cite{FroylandJungeKoltai2013}.

  However, for finite $N$, there is no guarantee that the reconstructed
  density $\tilde{\rho}(x,t) = \sum_{j=0}^N \rho_{N,j}(t) f_j(x)$
  is non-negative. 
  Generically $\tilde{\rho}(x,t)$ will take on both signs, in
  contrast with the exact solution $\rho(x,t)$ (see figure \ref{fig:transfer_op}).
  
  \begin{figure}[h!]
	\centering
  	\includegraphics[width=0.4\textwidth]{./images/transfer_op_t1_00.png}
  	\includegraphics[width=0.4\textwidth]{./images/transfer_op_t2_00.png}
	\caption{Red lines indicate numerically computed densities advected by, $\dot{x} = \sin(2x)$, via the transfer operator method with a resolution of $2\pi / 2^5$
		at time $t=1$ and $t=2$. The initial condition is a uniform distribution.
		Black lines plot the exact solution given by \eqref{eq:benchmark}.}
	\label{fig:transfer_op}
\end{figure}
  
  Moreover, important entities such as the advected moments
  $$\tilde{m}^k_i(t) = \int \tilde{\rho}(x,t) (\Phi_X^t)_*((x_i)^k)dx$$
  may fluctuate, also in contrast with the exact moments
  $$m^k_i = \int \rho(x,t)  (\Phi_X^t)_*((x_i)^k) dx.$$
  From the standpoint of interpretation, positivity and moments are important
  aspects of probability densities.  There is no agreed upon interpretation of
  ``negative probability'', and aspects such as moments serve as important
  qualitative characteristics of a given probability density (the zeroth moment
  is the average, the first moment is the variance, and so on).
  
  When dealing with manifolds of even moderate dimensions
  (e.g. $d=3$) it is important for a method to behave well at finite $N$'s
  because it is infeasible to finely resolve along each dimension.
  If the qualitative aspects of flows (such as conserved quantities and positivity
  constraints) can be incorporated into the numerics at arbitrary resolutions
  it is more likely that the resulting numerical scheme will produce qualitatively
  accurate results.
  This allows the analyst to focus on other aspects of the system at hand
  without worrying about the inconsistency errors which arise in schemes
  which do not preserve structure.

\subsection{Main contributions}
  In this paper we will present a method for advection of
  probability densities on manifolds.
  This method will yield reconstructed probability measures
  which are non-negative and mass conserving at any finite resolution.

\section{Mathematical preliminaries}
\label{sec:math}
  Throughout this section we will have the following
  setup.  Let $M$ be a smooth manifold.
  We denote the tangent bundle of $M$ by $TM$.
  Given any $C^1$ function $f:M \to N$,
  we denote the tangent lift of $f$ by $Tf:TM \to TN$.

\subsection{Densities}
  A smooth density on $M$ is a smooth means of
  assigning real numbers to measurable sets.
  Heuristically, it is a map which
  takes an infinitesimal box (or volume element)
  on a manifold as input, and outputs the infinitesimal ``size''
  of the box (a real number).
  Therefore, in order to discuss densities,
  we must first formalize the notion of an ``infinitesimal box.''
  This motivates our introduction of \emph{frames}.
  \begin{defn}
  \label{eq:frame_bundle}
    Given a manifold $M$, and a point $x \in M$,
    a \emph{frame at $x$} is a basis on the tangent space $T_x M$.
    We denote the set of frames at $x$ by $\Fr_x M$.
    The frame bundle is $\Fr M = \cup_{x \in M} \Fr_x M$.
  \end{defn}

  \begin{prop}
    There is a transitive
    action of $\GL(n)$ on each fiber of $\Fr M$.
  \end{prop}

  \begin{proof}
    Let $e = \{ e_1,\dots,e_n \} \in \Fr_x M$ for some $x \in M$.
    For each $A \in \GL(n)$ define the left action
    \begin{align*}
      A \cdot (e_1,\dots,e_n) := (A^j_1 e_j , \dots, A_n^j e_j ). 
    \end{align*}
    By inspection this actions is free and transitive.
    \footnote{This makes $\Fr M$ a $\GL(n)$ principal bundle over $M$.}
  \end{proof}

  Now that we understand frames (i.e. infinitesimal boxes)
  sufficiently well, we may introduce the notion of \emph{densities}.

  \begin{defn}[Appendix A \cite{BatesWeinstein1997}]
    \label{eq:density}
    Let $\alpha > 0$.
    An $\alpha$-\emph{density} on a manifold, $M$, is a map
    $\rho:\Fr M \to \R$ such that
    for any frame $e \in \Fr M$ and $A \in \GL(n)$,
    $
      \rho( A \cdot e ) =  | \det(A) |^\alpha \rho(e).
    $
    We denote the space of densities by $\Dens^\alpha(M)$.
    A $1$-density is simply called a \emph{density}, and
    so we denote $\Dens^1(M)$ by $\Dens(M)$.
    The integral of a density is defined via the same construction as that
    for $n$-forms \cite[Ch. 14]{Lee2006}.
    A \emph{mass density} is a density which is non-negative.
    and it is called a \emph{probability density} if its integral is
    unity.
  \end{defn}

  One can observe immediately that $\Dens^\alpha(M)$ is a vector-space.
  Despite this commonality with tensors,
  $1$-densities are \emph{not} tensors.
  Densities are very close in spirit to $n$-forms,
  but unlike $n$-forms, densities are \emph{non-oriented}
  due to the use of ``$|\det(A)|$'' rather than ``$\det(A)$'' 
  in the definition.
  Therefore, a density will not flip signs under a change of basis.
  This allows for the integral of a density to be well defined
  on non-orientable manifolds as well \cite[Ch. 14]{Lee2006}.

  \begin{prop}[Appendix A \cite{BatesWeinstein1997}]
    Let $\psi_1,\psi_2 \in \Dens^{1/2}(M)$.
    The function, $\psi_1 \psi_2$, obtained by
    scalar multiplication is a $1$-density.
    The pairing
    $
    \langle \psi_1, \psi_2 \rangle := \int_M \psi_1 \psi_2 
    $
    is a real inner-product.
    Finally, for any density $\rho$, the functions $\pm\sqrt{\rho}$ are
    $\frac{1}{2}$-densities.
  \end{prop}
  \begin{proof}
    Let $e \in \Fr M$ and $A \in \GL(n)$.
    We observe $\psi_1(A \cdot e) \psi_2(A \cdot e) = |\det(A) | \psi_1(e) \psi_2(e)$.  Thus $\psi_1 \psi_2 \in \Dens(M)$.
    Conversely $$\pm \sqrt{\rho( A \cdot e)} = \pm | \det(A) |^{1/2} \sqrt{ \rho(e)}.$$ So $\pm \sqrt{\rho} \in \Dens^{1/2}(M)$.
    Finally, if $\psi \neq 0$ we see that $\| \psi \|^2 := \langle \psi , \psi \rangle \neq 0$.
    Thus $\langle \cdot , \cdot \rangle$ is weakly non-degenerate
    and defines an inner-product on $\Dens^{1/2}(M)$.
  \end{proof}

  Note that for any $C^1$ diffeomorphism $\Phi:M \to M$,
  there is a map $\Fr(\Phi) : \Fr M \to \Fr M$
  given by 
  ``$(e_1,\dots,e_n) \mapsto (T\Phi \cdot e_1, \dots, T\Phi \cdot e_n)$''.
  This defines the \emph{pull-back} of an $\alpha$-density
  $\nu \in \Dens^\alpha(N)$
  by $\Phi^* \nu := \nu \circ \Fr(\Phi) \in \Dens^\alpha(M)$.
  \begin{prop} \label{prop:isom}
    Let $\Phi \in \Diff(M)$.
    The transformation ``$\psi \mapsto \Phi^* \psi$'' for
    $\psi \in \Dens^{1/2}(M)$ is an isometry with respect
    to the inner product on $\Dens^{1/2}(M)$.
  \end{prop}
  \begin{proof}
    Let $\psi_1,\psi_2 \in \Dens^{1/2}(M)$ and observe
    \begin{align*}
      \langle \Phi^* \psi_1, \Phi^* \psi_2 \rangle
      = \int \Phi^*( \psi_1 \psi_2)
      = \int \psi_1 \psi_2
      = \langle \psi_1, \psi_2 \rangle,
    \end{align*}
    where the equivalence of the integrals follows
    from \cite[Proposition 14.32(c)]{Lee2006}.
  \end{proof}
  
  Given a vector field $X \in \mathfrak{X}(M)$
  we can denote the flow by $\Phi^t_X \in \Diff(M)$
  and define the Lie-derivative of an alpha density by
  $
    \pounds_X[ \nu ] := \left. \frac{d}{dt} \right|_{t=0} (\Phi_X^t)^* \nu.
  $
  This yields the following corollary to proposition \ref{prop:isom}.
  \begin{cor}
    For any $X \in \mathfrak{X}(M)$, $\pounds_X[ {}\cdot{} ]$
    is an anti-symmetric linear operator on $\Dens^{1/2}(M)$.
  \end{cor}
  With the Lie-derivative defined we can write the advection
  PDE for $\alpha$-densities as
  \begin{align}
    \partial_t \nu + \pounds_X[\nu] = 0  \quad,\quad \nu(t) \in \Dens^\alpha(M) \label{eq:advection}.
  \end{align}
  This is the equation for a time-dependent $\alpha$-density which is advected by the vector field 
  $X \in \mathfrak{X}(M)$.
  If $\alpha = 1$, \eqref{eq:advection} is written in local coordinates
  as \eqref{eq:Louiville}.
  If $\nu(t) = \psi(t) \in \Dens^{1/2}(M)$ is a half-density, then
  \eqref{eq:advection} is written in local coordinates as
  \begin{align}
    \partial_t \psi(x) 
    + \frac{1}{2} X^i(x) \partial_i \psi(x) 
    + \frac{1}{2} \partial_i (\psi \cdot X^i)(x) \psi(x) = 0. \label{eq:local_half_density_advection}
  \end{align}
  Note that this equation appears to be the average of the advection equation for densities ($\Dens^1(M)$) and the advection equation for functions ($\Dens^0(M)$).

  \begin{thm} \label{thm:advection}
    Let $\rho(t) \in \Dens(M)$ be a time-dependent probability density
    and let $\psi \in \Dens^{1/2}(M)$ be such that $\rho = \psi^2$.
    Assume that $\rho$ is $C^1$.
    Let $X \in \mathfrak{X}(M)$.
    The following are equivalent:
    \begin{enumerate}
      \item $\rho$ satisfies the advection equation \eqref{eq:advection} with $\alpha = 1$ (locally given by \eqref{eq:Louiville}).
      \item $\psi$ satisfies the advection equation \eqref{eq:advection} with $\alpha = 1/2$ (locally given by \eqref{eq:local_half_density_advection}).
    \end{enumerate}
  \end{thm}
  \begin{proof}
    Let $\psi$ satisfy \eqref{eq:advection}.
    We find
    \begin{align*}
      \partial_t (\psi^2) &= 2 \partial_t\psi \cdot \psi
      =-2 \pounds_X[\psi] \psi 
      = - 2 \left[ \left.\frac{d}{dt}\right|_{t=0}
         (\Phi_X^t)^* \psi \right] \cdot \psi \\
      &= - \left. \frac{d}{dt} \right|_{t=0}
        \left[ (\Phi_X^t)^* \psi \cdot (\Phi_X^t)^* \psi \right]
      = - \left. \frac{d}{dt} \right|_{t=0}
        \left[ (\Phi_X^t)^* (\psi^2) \right] \\
      &= - \pounds_X[\psi^2].
    \end{align*}
    Therefore $\rho = \psi^2$ satisfies \eqref{eq:advection}.
    Conversely, if $\rho$ satisfies \eqref{eq:advection}
    and $\psi^2 = \rho$ then
    \begin{align*}
      \partial_t (\psi^2) = - \pounds_X[\psi^2] = - \pounds_X[\psi] \psi.
    \end{align*}
    Moreover, the right hand side is $2 (\partial_t \psi) \psi$.
    We can divide both side by $\psi$ at any point where $\psi(x) \neq 0$.
    By continuity of $\partial_t \rho$ and $\partial_t \psi$ we can
    verify \eqref{eq:advection} on the entire support of $\psi$
    (which is also the support of $\rho$).
    Outside the support it is neccesarily the case that
    $\rho = 0$ and  $\partial_i \rho = 0$.
    We observe that $\pounds_X[\rho] = 0$ as well.
    In this case $\pounds_X[\psi] = 0$ by the same argument.
    So we've verified \eqref{eq:advection} on the entire domain.
  \end{proof}

  We will use theorem \ref{thm:advection} later to justify building
  a numerical scheme to solve \eqref{eq:advection} with $\alpha = 1/2$
  in lieu of solving \eqref{eq:advection} with $\alpha = 1$.

\subsection{Euclidean realizations}
\label{sec:euclidean}
  We would like to apply wavelet theory later for the
  analytical tools and sparsity structure they carry.
  However, the notion of wavelets on manifolds is still young,
  and virtually all of the available wavelet analysis machinery
  is developed on Euclidean spaces and tori.
  In this section we will present theorems which allow
  us to transform analysis on manifolds into problems
  of analysis on subspaces of functions on $\R^n$.

\begin{thm}
  \label{thm:Euclidean}
  Let $\varphi:U \subset M \to V \subset \R^n$ be a chart.
  As $\varphi$ is injective, we can invert it on the range $V$.
  If $U$ is dense in $M$ then the maps
  \begin{align*}
    f \in C^k(M) &\mapsto \varphi_*f = f \circ \varphi^{-1} \in C^k(V) \\
    \nu \in \Dens^\alpha(M) &\mapsto \varphi_* \nu = \nu \circ \Fr(\varphi^{-1}) \in \Dens^\alpha(V) \\
    X \in \mathfrak{X}(M) &\mapsto \varphi_* X = T\varphi \cdot X \circ \varphi^{-1} \in \mathfrak{X}(V)
  \end{align*}
  are injective ring/vector-space/Lie-algebra morphisms respectively.
  \end{thm}
  \begin{proof}
    Let $f,g \in C^k(M)$ be such that $\varphi_* f = \varphi_*g$.
    Assume $f \neq g$.
    Since the range of $\varphi^{-1}$ is $U$, it must be the case that
    $f(x) \neq g(x)$ for some $x \notin U$.
    As $U$ is dense in $M$ there is a sequence $x_0,x_1,\dots$ in $U$
    which converges to $x$.
    As $f$ and $g$ are continous $g(x_i) = f(x_i)$ must converge to a
    unique limit.  Thus $g(x) = f(x)$, contradicting the assumption
    that $f \neq g$.
    Therefore the map $f \in C^k(M) \to \varphi_* f \in C^k(V)$
    is injective.
    That it is a ring morphism can then be viewed directly, $\varphi_*(f) \cdot \varphi_*(g) = \varphi_*(f \cdot g)$ for any $f,g \in C^k(M)$.
    
    The same argument applies to vector-fields
    upon noting $\varphi_*(X+cY) = \varphi_*X + c \varphi_*Y$ for any $X,Y \in \mathfrak{X}(M)$ and $c \in \R$, and $\varphi_*([X,Y]) = [\varphi_*X, \varphi_*Y]$.

    Finally, the same argument applies to $\alpha$-densities
    upon noting $\varphi_*( \nu + c \mu) = \varphi_*\nu + c \varphi_* \mu$ for any $\nu,\mu \in \Dens^\alpha(M)$.
  \end{proof}
  
  \begin{cor}
    Assume the setup of theorem \ref{thm:Euclidean}.
    The map 
    ``$\psi \in \Dens^{1/2}(M) \mapsto \varphi_* \psi \in \Dens^{1/2}(V)$''
    is an isometry.
  \end{cor}
  \begin{proof}
    Simply observe
    \begin{align*}
      \langle \varphi_*\psi_1 , \varphi_*\psi_2 \rangle
      = \int_V \varphi_*(\psi_1) \varphi_*(\psi_2) 
      = \int_V \varphi_*( \psi_1 \cdot \psi_2) 
      = \int_U \psi_1 \cdot \psi_2.
    \end{align*}
    As $U$ is dense in $M$, the above integral is unchanged by integration
    over $M$.   Thus we've verified $\langle \varphi_* \psi_1,\varphi_*\psi_2 \rangle = \langle \psi_1,\psi_2 \rangle$.
  \end{proof}

  The upshot of theorem \ref{thm:Euclidean} is that we may represent PDEs on manifolds
  as PDE's on Euclidean domains.
  In particular, theorem \ref{thm:Euclidean} equates the function
  space $C^k(M)$ with the subring
  \begin{align*}
    \varphi_*C^k(M) := \{ g \in C^k(V) \mid g = f \circ \varphi^{-1} , f \in C^k(M) \} \subset C^k(V).
  \end{align*}
  Thus any, PDE on $M$ can be fully represented as a PDE on a 
  subring of functions on $V$. 
  %\ram{This point isn't obvious enough to me. I think we are making three separate points. First, we are saying we can represent the evolution in $V$ using Euclidean methods, we can then use Theorem 2.8 to define the density on $U$. Second, we can define a unique extension to any function or density defined on a dense set of points on any Hausdorff space (this needs a citation and is it obvious that it extends to densities?). Finally, we want to say that for many (all?) manifolds such a dense chart exists and we can construct it explicitly for many examples like we do below. Am I right?}
  
  This is useful for solving the PDE under concern because it allows
  us to implement wavelet analysis on manifolds.
  Specifically, given our PDE on $\Dens^{1/2}(M)$,
  we may invoke the above isometry to get a PDE
  on the space $\Dens^{1/2}(V) \equiv L^2(V)$.
  We can then solve this PDE on $L^2(V)$ in a wavelet
  basis.  This will give us time dependent 
  function which we can pull-back to $M$.

% \begin{thm} \label{thm:Euclidean}
%   Let $\varphi : V \subset \R^n \to U \subset M$ be a chart on 
%   a compact $C^k$-manifold $M$ such that $U$ is dense in $M$.
%   We can extend $\varphi$ to a surjective map
%   $\bar{\varphi}: \bar{V} \to M$ such that
%   $M$ is $C^k$-diffeomorphic to $\bar{V} / \bar{\varphi}$.
% \end{thm}
% \begin{proof}
%   Let $\{x_k\}$ we a sequence in $V$ which converges to the boundary
%   of $\bar{V}$, then $\varphi(x_k)$ converges to a unique point in
%   $M \backslash U$ because $M$ is compact and $\varphi$ is continous.
%   This extends $\varphi$ to the boundary of $\bar{V}$.

%   Observe that $\bar{\varphi}( \bar{V} ) = \bar{U} = M$.
%   By construction $\bar{\varphi}$ is injective on $\bar{V} / \bar{\varphi}$.
%   Thus $\bar{\varphi}$ is a bijection from $\bar{V} / \bar{\varphi}$ to $M$.
  
%   We have shown equivalence as sets.
%   To obtain equivalence as topological spaces use
%   the quotient topology induced by $\bar{\varphi}$.
%   To obtain $C^k$-equivalence, use the quotient topology
%   induced by the $k$th order jet of $\varphi$
%   and continuously extend to $\bar{V}$.
%   \footnote{This entails interpreting the $k$th order Taylor expansions
%   of $\varphi$ as  a map to the space of multidimensional polynomials.
%   We then apply the same argument used to prove equivalence as topological
%   spaces to this map.}
% \end{proof}
  \subsection{Example: A Eucldean realization of $S^2$}
  Consider the $2$-sphere, $S^2 \subset \R^3$.
  If we use spherical coordinates we obtain a map
  $\varphi: (-\pi,\pi) \times (0,\pi) \to U \subset S^2$
  given by
  \begin{align*}
    \chi(\phi,\theta) = \begin{pmatrix}
      \cos(\phi) \sin(\theta) \\
      \sin(\phi) \sin(\theta) \\
      \cos(\theta)
      \end{pmatrix}.
  \end{align*}
  Then we find
  \begin{align*}
    &\varphi_*C^0(S^2) = \{
      f \in C^0( (-\pi,\pi)\times (0,\pi) \mid \\
    &\quad \begin{array}{l}
      \lim_{\theta \to 0} f(\phi,\theta) = f_{\rm north} \text{ for some } f_{\rm north} \in \R \\
      \lim_{\theta \to \pi} f(\phi,\theta) = f_{\rm south} \text{ for some } f_{\rm south} \in \R 
    \end{array}
    \}
  \end{align*}
  The canonical volume form on $S^2$ viewed as a subset of $\R^3$ is 
  given by
  \begin{align*}
    \mu_{S^2} = x dy \wedge dz - y dx \wedge dz + z dx \wedge dy.
  \end{align*}
  In spherical coordinates, this volume form is given by
  \begin{align*}
    \varphi_*(\mu_{S^2}) = \sin(\theta) d\theta \wedge d\phi.
  \end{align*}
  It is easy to observe that $|\mu_{S^2}|^\alpha \in \Dens^\alpha(M)$
  and arbitrary $\nu \in \Dens^\alpha(M)$ can always be written
  as $\nu = f_{\nu} \otimes |\mu_{S^2}|^\alpha$ for some function 
  $f_\nu \in C^0(M)$.
  The push-forward of $\nu$ by $\varphi$ is given by
  $\varphi_*\nu = (\varphi_*f_\nu) \otimes (\varphi_*|\mu_{S^2}|^\alpha)$.
  Therefore, by theorem \ref{thm:Euclidean},
  the inner-product space of half densities $\Dens^{1/2}(S^2)$
  is isometric to the subspace
  \begin{align*}
    \varphi_* \Dens^{1/2}(S^2) &= \left\{ \psi \in \Dens^{1/2}(V) \mid 
    	\begin{array}{l}
		\psi = f \otimes  (\sin(\theta) d\theta \wedge d\phi)^{1/2} , \\
		f \in \varphi_*C^0(S^2)
	\end{array}
	\right\}.
  \end{align*}

  % One concern is the assumption that there exists a chart which
  % is dense.  However, if $M$ is compact such a chart is guaranteed
  % to exist.

  % \begin{prop}
  % If $M$ is compact then there exists a chart whose domain is dense in $M$.
  % \end{prop}
  % \begin{proof}
  % Without loss of generality we will assume that $M$ is connected.
  % If $M$ is not connected we may apply the following argument to each component.
  % Equip $M$ with a Riemannian structure.
  % We may then invoke Lemma 4.4 of \cite{Sakai1996}.
  % This completes the proof, but we can provide details for the sake of 
  % completeness (pun intended).
  % For any $p \in M$ we can define the cut-locus by $C_p$ and the maximal open set $U_p \subset T_p M$ over which $\exp_p : T_p M \to M$ is injective.
  % We see that $U = \exp_p(U_p)$ is dense in $M$.
  % Upon choosing a basis, ${\bf e}_1,\dots,{\bf e}_n \in T_xM$,
  %  we obtain a chart $\varphi: V \subset \mathbb{R}^n \to U \subset M$.
  % Here $V$ is a star shaped subset of $\mathbb{R}^n$.
  % In particular, $\varphi(x) = \exp_p ( x^i {\bf e}_i )$.
  % \end{proof}

  % The existence of such a chart in the non-compact case is unclear
  % at this time.  More importantly,
  % although the proof is constructive (via Riemannian exponential maps)
  % this construction is not practical.
  % Typically the Riemannian exponential and the cut-locus are difficult
  % to compute. Despite this limitation,
  % we will find that a fairly large variety of manifolds can be handled.

  % For us, the most important aspects are the following
  % \begin{cor}\label{prop:functions_spaces}
  %   Let $M$ be a manifold with a dense chart 
  %   $\varphi:V \subset \R^n \to U \subset M$.
  %   The space $\Dens^\alpha(M)$ is isomorphic to the subspaces of 
  %   $\Dens^\alpha(\bar{V})$ given by
  %   \begin{align*}
  %     \bar{\varphi}^*\Dens^\alpha(M) := \{ \nu \circ \Fr(\bar{\varphi})
  %     \mid \nu \in \Dens^\alpha(M) \} \subset \Dens^\alpha( \bar{V}).
  %   \end{align*}
  %   The same construction yields $\mathfrak{X}(M)$ as isomorphic
  %   to a subalgebra 
  %   $\bar{\varphi}^* \mathfrak{X}(M) \subset \mathfrak{X}(\bar{V})$
  %   and $C^k(M)$ as isomorphic to a subring
  %   $\bar{\varphi}^* C^k(M) \subset C^k(\bar{V})$.
  % \end{cor}
  % \begin{proof}
  %   Recall that $\bar{\varphi}$ is surjective, but not injective
  %   on the boundary of $\bar{V}$.
  %   Let $\nu \in \Dens^\alpha(M), x \in V, e \in \Fr_x V, A \in \GL(n)$.
  %   We observe
  %   \begin{align*}
  %     \nu( A \cdot [ \Fr_x(\varphi) \cdot e ] ) =
  %     | \det(A) |^{\alpha} \nu( \Fr_x(\varphi) \cdot e)
  %   \end{align*}
  %   so that $\nu \circ \Fr(\varphi) \in \Dens^\alpha(V)$.
  %   By continous extension to $\bar{V}$ we find
  %   $\nu \circ \Fr(\bar{\varphi}) \in \Dens^\alpha(\bar{V})$.
  %   Let $\nu$ and $\mu$ be $\alpha$ densities,
  %   and let them differ on $M\backslash U$.
  %   Then $\mu$ and $\nu$ must also differ on $U$ 
  %   because $U$ is dense and $\nu$ and $\mu$ are continous.
  %   This would imply that $\varphi^*\nu$ and $\varphi^*\mu$ differ.
  %   Therefore, the map
  %   $\nu \in \Dens^\alpha(M) \mapsto \bar{\varphi}^*\nu \in \Dens^\alpha(\bar{V})$ is injective.
  %   By inspection this map is linear, so that the range is
  %   a subspace of $\Dens^\alpha(\bar{V})$.

  %   The same construction may be applied to vector-fields.
  %   Let $X \in \mathfrak{X}(M)$.
  %   We may restrict $X$ to $U$ and consider the pull-back
  %   $\varphi^*(X|_U) := T\varphi^{-1} \cdot X|_U \circ \varphi \in \mathfrak{X}(V)$.
  %   We may continously extend this to obtain a vector-field on $\bar{V}$.
  %   \footnote{This is despite the fact that $\bar{\varphi}^{-1}$
  %     does not exist.  We are merely using the fact that
  %   $\varphi^*(X|_U)$ is continous on $V$.}
  %   The same construction as before verifies injectivity.
  %   Since all maps involved are Lie algebra morphisms, we obtain
  %   a subalgebra of $\mathfrak{X}(\bar{V})$.

  %   The same construction proves $C^k(M)$ is a subring of $C^k(\bar{V})$
  %   via the ring morphism
  %   $f \in C^k(M) \mapsto f \circ \bar{\varphi} \in C^k(\bar{V})$.
  % \end{proof}


\section{An advection scheme}
\label{sec:scheme}
   In this section we will use Euclidean realizations of wavelet space
   to implement an advection scheme for densities on manifolds.
   More specifically, we want to solve the evolution equation
   \begin{align}
     \partial_t \rho + \pounds_X[\rho] = 0 \quad , \quad 
     \rho(0) = \rho_0 \in \Dens^1(M). \label{eq:density_advection2}
   \end{align}
   We are going to do this by first solving the half-density advection
   equation
   \begin{align}
     \partial_t \psi + \pounds_X[\psi] = 0 \quad , \quad
     \psi(0) = \psi_0 := \rho^{1/2} \in \Dens^{1/2}(M). \label{eq:half_density_advection}
   \end{align}
   If $\psi(t)$ is a solution to \eqref{eq:half_density_advection}, then theorem \ref{thm:advection} tells us that $\rho(t) := |\psi(t)|^2$
   is a solution to our original problem \eqref{eq:density_advection2}.
   Therefore we seek to solve \eqref{eq:half_density_advection}
   in lieu of solving \eqref{eq:density_advection2}.
   There are two reasons for doing this.
   Firstly, since $\rho(t)$ will be obtained by squaring something,
   it will neccessarily be positive.
   Secondly, we may invoke the natural inner-product on $\Dens^{1/2}(M)$
   to construct our advection scheme.
   There is no need to introduce a non-canonical innerproduct, as is
   the case in many spectral methods.
   
   The following is the algorithm we wish to consider in this paper.
   Let $E_n = \{ f_0, \dots, f_n \}$ be set of half-densities on $M$
   and let $\pr_n : \Dens^{1/2}(M) \to E_n$ be the orthogonal projection
   with respect to the inner-product on half-densities.
   For $X \in \mathfrak{X}(M)$, if we let $\pounds_X$ denote the Lie 
   derivative on $\Dens^{1/2}(M)$ then we can consider the operator
   \begin{align*}
     [\pounds_X]_n = \pr_n \circ \left. \pounds_X \right|_{E_n} : 
     E_n \to E_n
   \end{align*}
   as a finite-dimensional approximation of $\pounds_X$.
   If this approximation is any good
   (i.e. convergent as $n \to \infty$ a dense subspace of $\Dens^{1/2}(M)$)
   , we may solve the finite dimensional
   linear ODE
   \begin{align*}
     \dot{\psi}_n = [\pounds_X]_n \cdot \psi_n \quad , \quad
     \psi_n(0) = \pr_n( \psi_0 ) \in E_n.
   \end{align*}
   Then $\psi(t) \approx \psi_n(t)$ for sufficiently large $n$,
   and we have an approximate solution to \eqref{eq:half_density_advection},
   and therefore an approximation to \eqref{eq:density_advection2}.


%   If this approximation is any good, then it might be reasonable to assume
%   the flow is continous, and approximate the evolution of $\psi$ by
%   \begin{align*}
%     \psi^{(j)}(t) = e^{t [X]^{(j)} } \cdot \psi_0 \in E^j \quad \forall t.
%   \end{align*}
%   If $\psi(t)$ is the solution to \eqref{eq:advection2},
%   we are curious to bound the error
%   \begin{align*}
%     e = \| \psi(t) - \psi^{(j)}(t) \|_{L^2}.
%   \end{align*}
%   This is the content of theorem \ref{thm:convergence}.

% \subsection{Properties}
% \label{sec:properties}
% Some propeties:  Positivity,
% Anti-symmetry (diagnolizable, easily exponentiated),
% mass conserving.

% \hoj{Still need to write this section}


% \subsection{Convergence}
% \label{sec:convergence}

% In this section we will use the following assumption

% \begin{ass}
%   Both $\psi_\lambda$ and $\tilde{\psi}_\mu$ have $C^{\alpha +1}$
%   smoothness and are orthogonal to polynomials of degree $n \geq \alpha -1$.
% \end{ass}

% We begin by bounding functions to their orthogonal
% projections in $E^j$.

% \begin{lem}[see Theorem 3.10.2 or 3.2.2 of \cite{Cohen2003}]
%   \label{lem:polynomials}
%   Let $f \in H^n(V)$, and let $j \geq \frac{d}{2}$.
%   Then
%   \begin{align*}
%     \| f - {\rm Poly_j}(f) \|_{L^2} \leq 2^{-nj} \| f \|_{H^n}
%   \end{align*}
%   where ${\rm Poly}_j$
%   is the orthogonal projection of $L^2(V)$ onto polynomials
%   of degree $j$.
% \end{lem}

% The following Lemma is equation 4.6.4 of \cite{Cohen2003}
% with ``$r=1$''.

% \begin{lem} \label{lem:entry_bounds}
%   Let $[X]_{\lambda\mu} := \langle \tilde{\psi}_\mu, D_X[\psi_\lambda] \rangle_{L^2}$.  Then
%   \begin{align*}
%     | [X]_{\lambda \mu} |\leq C 2^{- \left(\frac{(d+1)}{2}+ \alpha \right)| |\lambda| - |\mu| | }
%     2^{( |\lambda | + |\mu|)/2}.
%   \end{align*}
%   where $C > 0$ depends on $X$ through the $C^k$-norm of the coefficients $a^\alpha$.
% \end{lem}

%   The following proof is taken from Example 1 in section 4.6 of \cite{Cohen2003}.
% \begin{proof}
%   By theorem 3.3.1 of \cite{Cohen2003} \todo{I don't really see how we are ausing Thm 3.3.1} and lemma \ref{lem:polynomials} we observe
%   \begin{align*}
%     | [X]_{\lambda \mu} | &\leq \inf_{g \in \Pi_n} \| D_X[\psi_\lambda] - g \|_{L^2( \supp(\tilde{\psi}_\mu) )} \\
%     &\leq C_1 2^{-|\mu|\frac{n}{2}}\| D_X[\psi_\lambda] \|_{L^\infty} \\
%     &\leq C_2 2^{-|\mu| \frac{n}{2}} \left( \sup_{\alpha} \| a^\alpha\|_{L^1 }  \right)
%     ( \| \psi_\lambda \|_{L^\infty} + \| \nabla \psi_\lambda \|_{L^\infty} )
%   \end{align*}
%   and by the definition of wavelets
%   \begin{align*}
%     &\leq C_3 \left( \sup_{\alpha} \| a^\alpha\|_{L^1 }  \right) 2^{(|\lambda|-|\mu|)n/2} (n+1) 2^{|\lambda|}.
%   \end{align*}
%   The desired result then follows by swapping $\mu$ and $\lambda$
%   and invoking the anti-symmetry of the operator $D_X$.
% \end{proof}

% \begin{lem} \label{lem:DX_is_bounded}
%   $D_X$ is a bounded linear operator from $H^n(V)$ to $L^2(V)$.
% \end{lem}
% \begin{proof}
%   The coefficients $a^m$ are assumed smooth and bounded.
%   So that $D_X [f] = a^m \partial_m f \in H^{n-1} \subset L^2$.
% \end{proof}

% \begin{lem}  \label{lem:L2_to_Hn}
%   Let $D_X^{(j)} := \pr_{E^j} \circ \left. D_X \right|_{E^j} : E^j \to E^j$.
%   There exists a $C>0$ such that for all $f \in H^s$
%   \begin{align}
%     \| (D_X - D_X^{(j)})[\pr_{E^j}(f)] \|_{L^2} \leq C 2^{-n(j+1)/2} \| f\|_{L^2}.
%   \end{align}
% \end{lem}
% \begin{proof}
%   Let $[D_X]_{\lambda\mu} = \langle \psi_{\mu} , D_X[\psi_\lambda] \rangle_{L^2}$
% and similarly for $[D_X^{(j)}]_{\lambda \mu}$.
% Note that if $|\lambda|>j$ or $|\mu |>j$ then $[D_X^{(j)}]_{\lambda\mu} = 0$.
% We find
%   \begin{align*}
%     &\| (D_X - D_X^{(j)}) [\pr_{E^j}(f))] \|^2_{L^2} = \\
%     &\quad \sum_{\lambda \in \Lambda, |\mu | \leq j} \left| [D_X]_{\lambda\mu} - [D_X^j]_{\lambda\mu} \right|^2 |[ \pr_{E^j}(f)]_\mu|^2 \\
%     &\quad = \sum_{|\mu|\leq j}| f_\mu |^2 \sum_{|\lambda| > j} | [D_X]_{\lambda\mu}|^2.
%   \end{align*}
%   By Lemma \ref{lem:entry_bounds} we see
%   \begin{align*}
%     &\leq C' \sum_{|\mu|\leq j} | f_\mu |^2  \sum_{|\lambda| > j } 
%     2^{- \left( \frac{d+1}{2} + \alpha \right) | |\lambda| - |\mu| |}
%     2^{ (|\lambda| + |\mu|)/2 }\\
%     &\leq \sum_{|\mu | \leq j , |\lambda| > j}
%     |f_\mu|^2 2^{ - \left( \frac{d}{2} + \alpha \right)|\lambda|} 
%     2^{|\mu|\left(\frac{d}{2} + 1 - \alpha\right) }
%   \end{align*}
%   The term $2^{|\mu|( d/2 + 1 - \alpha)}$ is bounded 
%   by $ 2^{j(d/2 + 1 - \alpha)}$,
%   Thus we find
%   \begin{align*}
%     &\leq C \|f\|_{L^2}^2 2^{j \left( \frac{d}{2} + 1 - \alpha\right)} \sum_{\lambda > j}
%       2^{-|\lambda|\left( \frac{d}{2} + \alpha\right)} \\
%     &\leq C' \|f\|_{L^2}^2 2^{j \left( \frac{d}{2} + 1 - \alpha\right)}
%     \frac{ 2^{ -\left(\frac{d}{2} + \alpha \right)}}{ 1 - 2^{-\left(\frac{d}{2} + \alpha\right)}} \\
%     &\leq C'' \|f\|_{L^2}^2 2^{-(j+1) \alpha}.
%   \end{align*}
% \end{proof}
%   \begin{cor} \label{cor:operator_bound}
%     \begin{align*}
%       &\| D_X [f] - D_X^{(j)} [\pr_{E^j}(f)] \|_{L^2} \leq \\
%       &\quad C \left( 2^{-nj} + 2^{-\alpha(j+1)/2} \right) \|f\|_{H^n}.
%     \end{align*}
%   \end{cor}
%   \begin{proof}
%     This follows directly from the inequality
%     \begin{align*}
%       &\| D_X [f] - D_X[\pr_{E^j}(f)] + D_X[\pr_{E^j}(f)] - D_X^{(j)}[\pr_{E^j}(f)] \|_{L^2} \\
%       &\quad \leq \|D_X\|_{L^2} \| f - \pr_{E^j}(f)\|_{L^2}
%       + \| (D_X - D_X^{(j)}) [\pr_{E^j}(f)] \|_{L^2}.
%     \end{align*}
%     and lemmas \ref{lem:DX_is_bounded} and \ref{lem:L2_to_Hn}.
%   \end{proof}

%   \begin{lem} \label{lem:flow_bound}
%     Let $A$ and $B$ be anti-Hermetian operators on a Hilbert space.
%     Let $x(t)$ satisfy $\dot{x} = Ax$ and $y(t)$ satisfy $\dot{y} = By$.
%     If $x(0) = y(0)$ and $\| A - B \| < \epsilon$, then $u(t) = \| x(t) - y(t) \|$ satisfies $u(t) < \epsilon t \| y(0) \|$
%   \end{lem}
 
%   \begin{proof}
%     Note that $\langle Ax,x\rangle = \langle By,y \rangle = 0$.
%     Using this we calculate
%     \begin{align*}
%       \frac{du}{dt} &= \frac{1}{2u(t)} \langle
%       Ax(t) - By(t) , x(t) - y(t) \rangle \\
%       &= \frac{1}{2u(t)} \langle
%       (A-B)y(t) , x(t) - y(t) \rangle \\
%       &\leq \frac{1}{2u(t)} \| A-B\|  \|y(t)\|  \|x(t) - y(t) \| \\
%       &< \epsilon \|y(t) \|
%     \end{align*}
%     However $\| y(t) \| = \| y(0) \|$ since $B$ is anti-Hermetian.
%     Thus $\frac{du}{dt} < \epsilon \| y(0) \|$.
%     The result follows by integration.
%   \end{proof}
%   The upshot of Lemma \ref{lem:flow_bound} is that we can bound the
%   operator $D_X$ to $D_X^{(j)}$ when restricted to $H^n$.
%   \begin{thm}\label{thm:convergence}
%     Let $\psi_0 \in L^2(V)$ be such that $|\psi_0|^2$ is a probability distribution.
%     Then the flows of $\dot{\psi} = D_X[\psi]$ and $\dot{\psi}^{(j)} = D_X^{(j)}[\psi^{(j)}]$ with initial condition $\psi_0$
%     satisfies
%     \begin{align*}
%       &\| \psi(t) - \psi^{(j)}(t) \|_{L^2}  \\
%       &\quad\leq t( 2^{-nj} + 2^{-n(j+1)/2}) \| \psi_0\|_{H^n} .
%     \end{align*}
%   \end{thm}
%   \begin{proof}
%     Combining corollary \ref{cor:operator_bound} with lemma \ref{lem:flow_bound}
%     yields
%     \begin{align*}
%       &\| \psi(t) - \psi^{(j)}(t) \|_{L^2}  \\
%       &\quad\leq t( 2^{-nj} + 2^{-n(j+1)/2}) \| \psi_0 \|_{H^n} \| \psi_0 \|_{L^2}.
%     \end{align*}
%     That $|\psi_0|^2$ is a probability distribution means $\| \psi_0\|_{L^2} = 1$.
%   \end{proof}


\section{A Benchmark computation}
Consider the ODE $X(x) = \sin(2x) \partial_x$ on the unit circle, $S^1$.
We can solve for the flow in closed from.  We find
$\Phi_X^t(x) = \arccot( e^{-2t} \cot(x) )$,
 and the inverse of $\Phi_X^t$ is the map $[\Phi_X^t]^{-1}(y) = \arccot( e^{2t} \cot(y) )$.
Using \eqref{eq:push_forward}
we obtain the solution to \eqref{eq:density_advection2} with initial condition $p_0(x)$ as the time-dependent density
\begin{align}
  p(x,t) = p_0\left[ \arccot \left( \cot(x)e^{2t} \right)\right]
  \left( e^{2t} \cos^2(x) + e^{-2t} \sin^2(x)  \right)^{-1}. \label{eq:benchmark}
\end{align}
Similarly, the solution to \eqref{eq:half_density_advection} with the initial condition $\psi_0(x)$ is the time-dependent half-density
\begin{align*}
  \psi(x,t) =  \quad \psi_0\left[\arccot\left( \cot(x)e^{2t} \right) \right]
  \left( e^{2t} \cos^2(x) + e^{-2t} \sin^2(x) \right)^{-1/2}. 
\end{align*}
We can use the Haar-wavelet to implement the method mentioned in section \ref{sec:naive}.
This equivalent to the transfer operator approach wherein one partitions the space into cells and computes fluxes \cite{FroylandJungeKoltai2013}.
We do this for a uniform distribution on the circle, and depict the numerically computed solutions at time $t=1,2$ in figure \ref{fig:transfer_op} (page \pageref{fig:transfer_op}).
This scheme appears to be at least qualitatively accurate at time $t=1$ and earlier.
However, the transfer operator method exhibits spurious spatial oscillations and negative probability densities at $t=2$ and beyond.

\begin{figure}[h!]
  \centering
  \includegraphics[width=0.4\textwidth]{./images/half_density_sqr_t1_00.png}
  \includegraphics[width=0.4\textwidth]{./images/half_density_sqr_t2_00.png}
  \caption{Snapshots of numerically computed densities at $t=1,2$ with respect to the ODE ``$\dot{x} = \sin(2x)$''.
  	Red indicates densities computed via our numerical scheme  (\S \ref{sec:scheme}) with a resolution of $2\pi / 2^5$
  	using the Haar wavelet. Black indicates the exact solution, \eqref{eq:benchmark}.
	The initial condition is a uniform density.}
  \label{fig:half_density_haar}
\end{figure}

For the purpose of comparison, we can also use the Haar wavelet to implement the half-density based method described in section \ref{sec:scheme}.
Numerically computed solutions are with respect to our new method are depicted in figure \ref{fig:half_density_haar} (page \pageref{fig:half_density_haar}).
We observe that the half-density advection scheme maintains positivity and the regularity of the exact solution.

Both schemes begin to fail once the exact solution becomes concentrated in a space below the chosen resolution of $2\pi \times 2^{-6}$.

Finally, we can also implement our own method using any wavelet. Snapshots of our method using two different Daubehcies (DB) wavelets (the DB 4 and 6 wavelets) each at two different scales are illustrated in figure \ref{fig:one_dim_system_wavelet} (page \pageref{fig:one_dim_system_wavelet}). Notice that although four alternatives appear accurate at the two time instances depicted, the DB 6 wavelet appears to perform at both scales. 

\begin{figure}[p]
  \centering
  \includegraphics[width=0.4\textwidth]{./images/S1Wavelet_sf4nBases88scale1T1.pdf}
  \includegraphics[width=0.4\textwidth]{./images/S1Wavelet_sf4nBases88scale1T2.pdf}
  \includegraphics[width=0.4\textwidth]{./images/S1Wavelet_sf4nBases160scale2T1.pdf}
  \includegraphics[width=0.4\textwidth]{./images/S1Wavelet_sf4nBases160scale2T2.pdf}
  \includegraphics[width=0.4\textwidth]{./images/S1Wavelet_sf6nBases100scale1T1.pdf}
  \includegraphics[width=0.4\textwidth]{./images/S1Wavelet_sf6nBases100scale1T2.pdf}
  \includegraphics[width=0.4\textwidth]{./images/S1Wavelet_sf6nBases176scale2T1.pdf}
  \includegraphics[width=0.4\textwidth]{./images/S1Wavelet_sf6nBases176scale2T2.pdf}
  \caption{Ground truth densities (black) and densities computed using our method (red) with different DB wavelets (rows $1$ and $2$ use the DB $4$ wavelet and rows $3$ and $4$ use the DB $6$ wavelet) at different scales (rows $1$ and $3$ are computed via bases functions with a resolution of $2^{-1}$ and rows $2$ and $4$ are computed via bases functions with a maximum resolution of $2^{-2}$) for the vector field $\dot{x}=\sin(2x)$ at $t=1$ (left column) and $2$ (right column) and with a uniform density at $t=0$. Beginning from the top row the number of bases functions used to compute the densities in each row are $88,160,100,$ and $176$.}
  \label{fig:one_dim_system_wavelet}
\end{figure}
\pagebreak
%\subsection{The Van der Pol oscillator}
%Consider the system on $\R^2$ given by the ODE
%\begin{align}
%  \begin{split}
%    \dot{x} &= y \\
%    \dot{y} &= (1-x^2)y-x
%   \end{split} \label{eq:vdp}
%\end{align}
%The vector-field is depicted in figure \ref{fig:vdp_quiver}.
%\begin{figure}[p]
%  \centering
%  \includegraphics[width=0.5\textwidth]{./images/vdp_traj.png}
%  \caption{A quiver plot of the vector-field given by \eqref{eq:vdp}.  The trajectory with initial condition (1.5,0) is shown in blue.}
%  \label{fig:vdp_quiver}
%\end{figure}
%
%This vector field is particularly difficult for multiple reasons.
%Firstly, the system is on a non-compact domain, so we are unable to
%resolve the full domain using a finite set of basis functions.
%Secondly, if we restrict ourselves to a compact subset, the boundary conditions are non-vanishing.  Even worse, they are a mixture of inflowing
%and outflowing.
%Finally, the system exhibits a limit cycle.
%As densities are attracted towards the limit cycle, they are 
%exponentially flattened.  Eventually the density will flatten 
%into a length scale below that which we've resolved.
%
%Despite these problems on the boundaries, and with the resolution,
%we are still able to construct a scheme which works on small times
%in regions near the origin.
%We consider a basis generated by DB wavelets,
%on the domain $[-8,8] \times [-8,8]$ with finest scale $2^{-3}$.
%A typical advection is shown in 
%figure \ref{fig:vdp_advection}.
%One can see the density begin to hug the side of the limit cycle
%at time $t=1$.
%
%
%\begin{figure}[p]
%  \centering
%  \includegraphics[width=0.43\textwidth]{./images/vdp_t0p0.png}
%  \includegraphics[width=0.43\textwidth]{./images/vdp_t0p30.png}
%  \includegraphics[width=0.43\textwidth]{./images/vdp_t0p64.png}
%  \includegraphics[width=0.43\textwidth]{./images/vdp_t0p97.png}
%  \caption{Computed densities given a Gaussian initial condition
%    and advected by \eqref{eq:vdp}.
%    Snapshots are displayed at time $t=0.00,0.30,0.64,0.97$.
%  }
%  \label{fig:vdp_advection}
%\end{figure}

\section{Example: the rigid body equations}
In the absence of external forces
the equations of motion for the angular momentum, $\Pi \in \R^3$,
of a rigid body are
\begin{align}
  \dot{\Pi} = \Pi \times ( \mathbb{I}^{-1}\cdot \Pi),  \label{eq:rigid_body}
\end{align}
where $\mathbb{I} = {\rm diag}(I_1,I_2,I_3)$,
and $I_1,I_2,I_3 > 0$ are rotational inertias along
the principal axes of rotation.
As $\dot{\Pi}$ is orthogonal to $\Pi$, we observe that
$\| \Pi \|$ is constant.  Thus the dynamics are constrained
to spheres.  Moreover, the dynamics on a sphere of radius $r>0$
are identical to the dynamics on a sphere of unit radius
upon rescaling time by $r^{-2}$.
Therefore we may (literally) restrict our analysis of this system
to an ode on the unit sphere $S^2 \subset \R^3$.
In spherical coordinates, the dynamics are given by
\begin{align*}
  \dot{\phi} &= -\sin^2(\phi) \cos(\theta) \frac{I_3-I_2}{I_3I_2}
  + \cos^2(\phi) \cos(\theta) \frac{I_1 - I_3}{I_1 I_3} \\
  \dot{\theta} &= - \sin(\theta) \sin(\phi) \cos(\phi) \frac{I_2 - I_1}{I_2I_1} \\
  \dot{r} &= 0
\end{align*}
A plot of various trajectories on $S^2$ is depicted in figure \ref{fig:rigid_body_traj}.

\begin{figure}[h]
  \centering
  \includegraphics[width=0.8\textwidth]{./images/rigid_body_traj.png}
  \caption{trajectories of \eqref{eq:rigid_body} on the unit sphere.}
  \label{fig:rigid_body_traj}
\end{figure}
We can consider the global chart induced by spherical coordinates.
Specifically, this is the chart with
\begin{align*}
  U = S^2 \backslash (0,0,-1) \quad,\quad
  V = (0,\pi) \times (-\pi,\pi) \quad,\quad
  \varphi(\theta,\phi) = \begin{pmatrix}
    \cos(\phi) \sin(\theta) \\
    \sin(\phi) \sin(\theta) \\
    \cos(\theta)
    \end{pmatrix}.
\end{align*}
We may approximate $C^0(V)$ using the Haar wavelet,
and then push-forward these approximations
to obtain an advection scheme on $S^2$.


\begin{figure}[h]
  \centering
  \includegraphics[width=0.24\textwidth]{./images/sphere_t0}
  \includegraphics[width=0.24\textwidth]{./images/sphere_t2p5}
  \includegraphics[width=0.24\textwidth]{./images/sphere_t5}
  \includegraphics[width=0.24\textwidth]{./images/sphere_t7p5}
  \caption{Snapshots at time $t=0.0,2.5,5.0,7.5$ of evolution on the 2-sphere, according to the rigid body equations.}
  \label{fig:sphere}
\end{figure}

We simulate an initial density which consists of 
a smooth bump function around the point $(-1,0,0)$.
There is a saddle point at $(-1,0,0)$, and so 
we should expect the distribution to tend towards
a singular distribution concentrated along the unstable
submanifold associated to the saddle point.
The result is depicted in figure \ref{fig:sphere}



\section{Conclusion and future work}
In this article we have presented a spectral advection scheme which preserves the positivity of probability densities at arbitrarily low resolutions.
However, there are two major items which will constitute future work
for clarifying the range of applicability of this scheme.
Firstly, we have not shown under which circumstance the scheme is
 convergent.
It may be intuitively clear that this scheme will converge
under a ``reasonable'' choice of basis, but this is not precise.
Future work entails clarifying what ``reasonable'' means.
Secondly, the scheme appears to conserve a number of other
properties at arbitrarily low resolutions.
For example, the space of functions is discretized as
a ring of Hermitian operators.
This ring structure is preserved under our scheme.
Enumerating the many other structures which are preserved 
by the scheme will be elucidated in future work as well.


\subsection{Acknowledgements}
H.O.J. is supported by European Research Council Advanced Grant 267382 FCCA.


\bibliographystyle{IEEEtran}
%\bibliography{./hoj_2014}
% Generated by IEEEtran.bst, version: 1.13 (2008/09/30)
\begin{thebibliography}{1}
\providecommand{\url}[1]{#1}
\csname url@samestyle\endcsname
\providecommand{\newblock}{\relax}
\providecommand{\bibinfo}[2]{#2}
\providecommand{\BIBentrySTDinterwordspacing}{\spaceskip=0pt\relax}
\providecommand{\BIBentryALTinterwordstretchfactor}{4}
\providecommand{\BIBentryALTinterwordspacing}{\spaceskip=\fontdimen2\font plus
\BIBentryALTinterwordstretchfactor\fontdimen3\font minus
  \fontdimen4\font\relax}
\providecommand{\BIBforeignlanguage}[2]{{%
\expandafter\ifx\csname l@#1\endcsname\relax
\typeout{** WARNING: IEEEtran.bst: No hyphenation pattern has been}%
\typeout{** loaded for the language `#1'. Using the pattern for}%
\typeout{** the default language instead.}%
\else
\language=\csname l@#1\endcsname
\fi
#2}}
\providecommand{\BIBdecl}{\relax}
\BIBdecl

\bibitem{LasotaMackey1994}
A.~Lasota and M.~C. Mackey, \emph{Chaos, Fractals, and Noise}, ser. Applied
  Mathematical Sciences.\hskip 1em plus 0.5em minus 0.4em\relax Springer
  Verlag, 1994.

\bibitem{FroylandJungeKoltai2013}
\BIBentryALTinterwordspacing
G.~Froyland, O.~Junge, and P.~Koltai, ``Estimating long-term behavior of flows
  without trajectory integration: the infinitesimal generator approach,''
  \emph{SIAM J. Numer. Anal.}, vol.~51, no.~1, pp. 223--247, 2013. [Online].
  Available: \url{http://dx.doi.org/10.1137/110819986}
\BIBentrySTDinterwordspacing

\bibitem{BatesWeinstein1997}
S.~Bates and A.~Weinstein, \emph{Lectures on the geometry of quantization},
  ser. Berkeley Mathematics Lecture Notes.\hskip 1em plus 0.5em minus
  0.4em\relax Providence, RI: American Mathematical Society, 1997, vol.~8.

\bibitem{Lee2006}
J.~M. Lee, \emph{Introduction to smooth manifolds}, ser. Graduate Texts in
  Mathematics.\hskip 1em plus 0.5em minus 0.4em\relax Springer-Verlag, 2006,
  vol. 218.

\end{thebibliography}
\end{document}
