\documentclass[12pt]{amsart}
\usepackage{amsmath,amssymb}
\usepackage{geometry} % see geometry.pdf on how to lay out the page. There's lots.
\geometry{a4paper} % or letter or a5paper or ... etc
% \geometry{landscape} % rotated page geometry

%  POSSIBLY USEFULE PACKAGES
%\usepackage{graphicx}
%\usepackage{tensor}
%\usepackage{todonotes}

%  NEW COMMANDS
\newcommand{\pder}[2]{\ensuremath{\frac{ \partial #1}{\partial #2}}}
\newcommand{\ppder}[3]{\ensuremath{\frac{\partial^2 #1}{\partial
      #2 \partial #3} } }

%  NEW THEOREM ENVIRONMENTS
\newtheorem{thm}{Theorem}[section]
\newtheorem{prop}[thm]{Proposition}
\newtheorem{cor}[thm]{Corollary}
\newtheorem{lem}[thm]{Lemma}
\newtheorem{defn}[thm]{Definition}


%  MATH OPERATORS
\DeclareMathOperator{\Diff}{Diff}
\DeclareMathOperator{\GL}{GL}
\DeclareMathOperator{\SO}{SO}
\DeclareMathOperator{\ad}{ad}
\DeclareMathOperator{\Ad}{Ad}
\DeclareMathOperator{\acot}{acot}

%  TITLE, AUTHOR, DATE
\title{Solution to Louiville equation for $\dot{x} = \sin(2x)$}
\author{Henry O. Jacobs}
\date{\today}


\begin{document}

\maketitle

Consider the vector field $\dot{x} = \sin(2x)$ for $x \in S^1$ where
$S^1$ is the unit-circle.

\begin{prop}
  The solution to this ode $\frac{dx}{dt} = \sin(2x)$
  with initial condition $x_0 \in S^1$ is given by
  \begin{align*}
  x(t) = \Phi^t(x_0)
\end{align*}
where $\Phi^t$ is given (for $x_0 \in (0,2\pi)$) by the flow map
\begin{align*}
  \Phi^t(x_0) = \acot( e^{-2t} \cot(x_0) ).
\end{align*}
\end{prop}
\begin{proof}
  Let $x(t) = \Phi^t(x_0) =\acot(e^{-2t}\cot(x_0))$.
  By inspection we can invert this equation to find that
  $x_0 = \acot( e^{2t} \cot(x(t) ) )$.
  We observe directly that $x(0) = x_0$.
  Let $y(t) = e^{-2t} \cot(x_0)$ so that we may write $x(t) = \acot(y(t))$.
  Note that $dy/dt = -2y$.
  By the chain rule we find
  \begin{align*}
    \frac{dx}{dt} &= \left.
      \left( \frac{d}{dy}( \acot(y) ) \frac{dy}{dt} \right)
      \right|_{y= e^{-2t} \cot(x_0) } \\
      &= \frac{-1}{1 + y^2} \cdot -2 y \\
      &= \frac{2e^{-2t} \cot( x_0)}{1+( e^{-2t} \cot(x_0) )^2 }
  \end{align*}
  Substitute the identity $x_0 = \acot( e^{2t} \cot(x) )$ into the above to find
  \begin{align*}
    \frac{dx}{dt} &= \frac{2  \cot(x) }{ 1+ \cot^2(x) } \\
    &= 2 \sin(x) \cos(x) = \sin(2x)
  \end{align*}
  which completes the proof.
\end{proof}


As described in the above proof,
the inverse of the flow, $\Phi^t$, is given by
\begin{align*}
  (\Phi^t)^{-1}(y) = \acot( e^{2t} \cot(y) ).
\end{align*}

Let $p_0$ be a non-negative function and $dx$ be the uniform measure on $S^1$.
Then $p_0 dx$ is a density.
The formula in our paper states that 
\begin{align*}
  (\Phi^t)_*( p_0 dx)(x) := p_0 ((\Phi^t)^{-1}(x)) \frac{d (\Phi^t)^{-1}}{dx}|_{x} dx
\end{align*}

We observe that
\begin{align*}
  \frac{d (\Phi^t)^{-1}}{dx}|_x &= \frac{d}{dx} \left( \acot( e^{2t} \cot(x) ) \right) = \frac{-1}{1+ (e^{2t} \cot(x))^2} \cdot \frac{d}{dx} \left( e^{2t}   \cot(x) \right) \\
  &= \frac{-1}{1 + e^{4t} \cot^2(x)} e^{2t} \frac{1}{\sin^2(x)} \\
  &= \frac{1}{ e^{-2t} \sin^2(x) + e^{2t} \cos^2(x) }.
\end{align*}

Thus we find
\begin{align*}
  (\Phi^t)_* ( p_0 dx) (x) = \frac{p_0( \acot( e^{2t} \cot(x) ) ) dx}{
    e^{-2t} \sin^2(x) + e^{2t} \cos^2(x) }.
\end{align*}


\bibliographystyle{amsalpha}
\bibliography{/Users/hoj201/Dropbox/hoj_2014.bib}
\end{document}
